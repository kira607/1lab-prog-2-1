\section*{Формулировка задания}

Разработать программу, обрабатывающую текстовую информацию,
представленную массивом указателей на строки (массивы символов).
Программа должна считать данные (текстовую информацию) из
входного файла, представленного набором строк (каждая строка
представляет собой последовательность символов, среди которых могут быть
буквы, пробелы, знаки препинания и т.п.). 

Далее над текстом выполняется
операция, определённая в вариантах задания. 
После чего результаты
выполнения операции должны быть отражены на экране и записаны в новый
файл.

Доступ к каждой строке массива, а также к отдельному элементу
(символу) строки должно осуществляться через указатели.

\textbf{Индивидуальный вариант:} Удалить все слова, 
содержащие заданный символ, в каждой строке текста.