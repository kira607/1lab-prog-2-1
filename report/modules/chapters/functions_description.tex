\section*{Описание пользовательских функций}

\subsection*{DeleteWordsWithChar}

\begin{lstlisting}
	void DeleteWordsWithChar(char *strings[], int lines_number, char bad_char);
\end{lstlisting}

Функция по удалению слов, содержащих заданный символ, из массива строк. 
Принимает указатель на массив строк, 
количество строк и символ, который должен содержаться в словах для удаления.

Алгоритм работы функции:

\begin{itemize}
	\item Цикл по \verb|line| от \verb|0| до \verb|lines_number|
		\begin{itemize}
			\item \verb|char *string = strings[line];|
			\item \verb|char *new_string = (char*)calloc(sizeof(char), strlen(string));|
			\item \verb|char *word = (char*)calloc(sizeof(char), strlen(string));|
			\item Если \verb|!IfStringContainsChar(string, bad_char)|
				\begin{itemize}
					\item \verb|continue;|
				\end{itemize}
			\item Цикл по \verb|character_pos| от \verb|0| до \verb|strlen(string)|
				\begin{itemize}
					\item \verb|char character = string[character_pos];|
					\item Если \verb|character == ' ' |||
							   \verb| character == '\n' |||
							   \verb| character == EOF |||
							   \verb| character == '\0'|
						\begin{itemize}
							\item Если \verb|!IfStringContainsChar(word, bad_char)|
							\item \verb|    strcat(new_string, word);|
							\item \verb|    strcat(new_string, character);|
							\item \verb|free(word);|
							\item \verb|word = (char*)calloc(sizeof(char), strlen(string));|
							\item \verb|continue;|
						\end{itemize}
					\item Иначе
						\begin{itemize}
							\item \verb|strcat(word, character);|
						\end{itemize}
				\end{itemize}
			\item \verb|free(strings[line]);|
			\item Если \verb|new_string[strlen(new_string) - 1] != '\n'|
				\begin{itemize}
					\item \verb|strcat(new_string, '\n');|
				\end{itemize}
			\item \verb|strings[line] = new_string;|
		\end{itemize}
\end{itemize}

\subsection*{ReadFromFile}

\begin{lstlisting}
	int ReadFromFile(const std::string& file_name, char **&strings);
\end{lstlisting}

Функция для чтения строк из файла. 

Принимает имя файла и 
ссылку на указатель на указатель на char 
(необходимо, чтобы модифицировать оригинальный указатель). 

Возвращает количество прочитанных строк.

Алгоритм работы функции:

\begin{itemize}
    \item \verb|FILE *input_file;|
	\item Если \verb|(input_file = fopen(file_name.c_str(), "r")) == nullptr|
		\begin{itemize}
			\item \verb|std::cout << "Could not open \"" << file_name << "\"\n";|
			\item \verb|return -1;|
		\end{itemize}
    \item \verb|const int string_chunk_len = 10, lines_chunk_len = 3;|
    \item \verb|int line = 0;|
    \item \verb|int line_size = 0;|
    \item \verb|strings = (char**)malloc(sizeof(char*) * lines_chunk_len);|
    \item \verb|strings[0] = (char*)malloc(sizeof(char) * string_chunk_len);|
    \item \verb|char current_char;|
	\item Пока \verb|(current_char = static_cast<char>(fgetc(input_file))) != EOF|
		\begin{itemize}
			\item \verb|++line_size;|
			\item Если \verb|line_size % string_chunk_len == 0|
				\begin{itemize}
					\item \verb|strings[line] = |\\
					\verb|(char*)realloc(strings[line], sizeof(char) *|\\
					\verb|(line_size + string_chunk_len));|
				\end{itemize}
			\item \verb|strcat(strings[line], current_char);|
			\item Если \verb|current_char == '\n'|
				\begin{itemize}
					\item \verb|strcat(strings[line], '\0');|
					\item \verb|line_size -= line_size;|
					\item \verb|++line;|
					\item Если \verb|line % lines_chunk_len == 0|
					\subitem \verb|strings = |\\
					\verb|(char**)realloc(strings, sizeof(char*) *|\\
					\verb|(line + lines_chunk_len));|
					\item \verb|strings[line] = (char*)calloc(string_chunk_len, sizeof(char));|
				\end{itemize}
		\end{itemize}
    \item \verb|fclose(input_file);|
    \item \verb|return line_size ? line+1 : line;|
\end{itemize}

\subsection*{WriteInFile}

\begin{lstlisting}
	void WriteInFile(const std::string& file_name, char **original_strings, char **strings, char bad_char, int lines_number)
\end{lstlisting}

Функция для записи массива строк в файл.

Принимает указатель на массив оригинальных строк, 
указатель на массив изменённых строк, символ, 
который содержался в удалённых словах и количество строк.

Алгоритм работы функции:

\begin{itemize}
	\item \verb|FILE* output_file = nullptr;|
    \item Если \verb|(output_file = fopen(file_name.c_str(), "w")) == nullptr|
    \item \verb|    std::cout << "Could not open \"" << file_name << "\"\n";|
    \item \verb|    return;|
	\item Цикл по \verb|line| от \verb|0| до \verb|lines_number|
		\begin{itemize}
			\item \verb|fputs(original_strings[line], output_file);|
		\end{itemize}
    \item \verb|fputs("\n\nOperation: Delete words with a specific character\n", output_file);|
    \item \verb|fputs("Parameters: ", output_file);|
    \item \verb|fputc(bad_char, output_file);|
    \item \verb|fputs("\n\n", output_file);|
	\item Цикл по \verb|line| от \verb|0| до \verb|lines_number|
		\begin{itemize}
			\item \verb|fputs(strings[line], output_file);|
		\end{itemize}
    \item \verb|fclose(output_file);|
\end{itemize}

\subsection*{IfStringContainsChar}

\begin{lstlisting}
	bool IfStringContainsChar(char* string, char bad_char)
\end{lstlisting}

Функция для проверки содержания в строке заданного символа.

Принимает указатель на строку и проверяемый символ.

Возвращает \textbf{true}, если символ найден, \textbf{false} в противном случае.

Алгоритм работы функции:

\begin{itemize}
    \item \verb|char *word_ptr = string;|
	\item Пока \verb|*word_ptr|
		\begin{itemize}
			\item Если \verb|strchr(&bad_char, *word_ptr)|
			\subitem \verb|return true;|
			\item \verb|++word_ptr;|
		\end{itemize}
    \item \verb|return false;|
\end{itemize}

\subsection*{PrintArrayOfStrings}

\begin{lstlisting}
	void PrintArrayOfStrings(char **strings, int lines\_number);
\end{lstlisting}

Функция для печати массива строк.

Принимает указатель на массив строк и количество строк.

Алгоритм работы функции:

\begin{itemize}
	\item Цикл по \verb|line| от \verb|0| до \verb|lines_number|
	\begin{itemize}
		\item \verb|printf("%s", strings[line]);|
	\end{itemize}
\end{itemize}

\subsection*{strcat}

\begin{lstlisting}
	char *strcat (char *dest, char character);
\end{lstlisting}

Перегрузка функции strcat из strings.h.

Принимает указатель на исходную строку и символ для добавления в конец.

Возвращает указатель на новую строку с добавленным символом.

Алгоритм работы функции:

\begin{itemize}
	\item \verb|char* char_ptr = (char*)malloc(sizeof(char)*2);|
    \item \verb|*char_ptr = character;|
    \item \verb|*(char_ptr + 1) = '\0';|
    \item \verb|strcat(dest, char_ptr);|
    \item \verb|free(char_ptr);|
    \item \verb|return dest;|
\end{itemize}
