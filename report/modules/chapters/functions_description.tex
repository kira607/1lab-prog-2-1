\section*{Описание пользовательских функций}

\subsection*{DeleteWordsWithChar}
void DeleteWordsWithChar(char *strings[], int lines\_number, char bad\_char);

Функция по удалению слов, содержащих заданный символ, из массива строк. Принимает указатель на массив строк, количество строк и символ, который должен содержаться в словах для удаления.

Алгоритм работы функции:

Цикл по line от 0 до lines\_number
	\begin{itemize}
		\item Записываем указатель на текущую строку в string
		\item Выделяем память под новую строку и записываем указатель в new\_string
		\item Выделяем память под слово и записываем указатель в word
		\item Если IfStringContainsChar для текущей строки возвращает \textbf{false}, сбрасываем текущую итерацию цикла
		\item Цикл по character\_pos от 0 до длины string
		\begin{itemize}
			\item Записываем текущий символ в character
			\item Если character == ' ' или character == '\textbackslash n' или character == EOF или character == '\textbackslash 0'
			\begin{itemize}
				\item Если IfStringContainsChar возвращает \textbf{true} для word
				\subitem strcat(new\_string, word)
				\subitem strcat(new\_string, character)
				\item Освободить указатель word
				\item Выделить память под слово и записать указатель в word
				\item Сбросить текущую итерацию цикла
			\end{itemize}
			\item Иначе
			\subitem strcat(word, character);
		\end{itemize}
	\item Освобождаем указатель на проверенную строку
	\item Если послений символ строки не \textbackslash n, добавляем \textbackslash n в конец
	\item Присваиваем указателю адрес ячейки памяти, на которую указывает new\_string
	\end{itemize}

\subsection*{ReadFromFile}
int ReadFromFile(const std::string\& file\_name, char **\&strings);

Функция для чтения строк из файла. 

Принимает имя файла и ссылку на указатель на указатель на char (необходимо, чтобы модифицировать оригинальный указатель). 

Возвращает количество прочитанных строк.

Алгоритм работы функции:

\begin{itemize}
	\item Пытемся открыть файл. В случаем провала возвращаем -1
	\item Задаём константы string\_chunk\_len = 10, lines\_chunk\_len = 3
	\item Выделяем память под массив строк длинной lines\_chunk\_len
	\item Выделяем память под нулевую строку длинной string\_chunk\_len
	\item Создаём место под считываемый символ
	\item Пока считанный из файла символ в current\_char не равен EOF
	\begin{itemize}
		\item ++line\_size
		\item Если длина строки line\_size кратна string\_chunk\_len, расширяем текущую строку на string\_chunk\_len
		\item Записываем считанный символ в текущую строку.
		\item Если считанный символ == \textbackslash n
		\begin{itemize}
			\item Записываем символ конца строки в текущую строку
			\item line\_size = 0
			\item ++line
			\item Если размер массива line кратен lines\_chunk\_len, расширяем массив строк на lines\_chunk\_len
			\item Выделяем память под нулевую строку длинной string\_chunk\_len
		\end{itemize}
	\end{itemize}
	\item Закрываем файл
	\item Возвращаем количество считанных строк
\end{itemize}

\subsection*{WriteInFile}
void WriteInFile(const std::string\&, char **, char **, char, int);

Функция для записи массива строк в файл.

Принимает указатель на массив оригинальных строк, указатель на массив изменённых строк, символ, который содержался в удалённых словах и количество строк.

Алгоритм работы функции:

\begin{itemize}
	\item Пытемся открыть файл. В случаем провала возвращаемся
	\item Записываем в файл оригинальные строки
	\item Записываем в файл совершённую операцию
	\item Записываем в файл параметры операции
	\item Записываем в файл модифицированные строки
	\item Закрываем файл
\end{itemize}

\subsection*{IfStringContainsChar}
bool IfStringContainsChar(char* string, char bad\_char);

Функция для проверки содержания в строке заданного символа.

Принимает указатель на строку и проверяемый символ.

Возвращает true, если символ найден, false в противном случае.

Алгоритм работы функции:

\begin{itemize}
	\item Записываем указатель на строку в word\_ptr
	\item Пока значение word\_ptr не 0
	\begin{itemize}
		\item Если strchr(\&bad\_char, *word\_ptr) Возвращаем \textbf{true}
		\item ++word\_ptr
	\end{itemize}
	\item Возвращаем \textbf{false}
\end{itemize}

\subsection*{PrintArrayOfStrings}
void PrintArrayOfStrings(char **strings, int lines\_number);

Функция для печати массива строк.

Принимает указатель на массив строк и количество строк.

Алгоритм работы функции:

\begin{itemize}
	\item Циклом по line от 0 до lines\_number печатаем строку
\end{itemize}

\subsection*{strcat}
char *strcat (char *dest, char character);

Перегрузка функции strcat из strings.h.

Принимает указатель на исходную строку и символ для добавления в конец.

Возвращает указатель на новую строку с добавленным символом.

Алгоритм работы функции:

\begin{itemize}
	\item Выделяем память под 2 символа (char\_ptr)
	\item В первую ячейку записываем character
	\item Во вторую ячейку записываем \textbackslash 0
	\item strcat(dest, char\_ptr)
	\item Освобождаем память char\_ptr
	\item Возвращаем указатель на dest
\end{itemize}
